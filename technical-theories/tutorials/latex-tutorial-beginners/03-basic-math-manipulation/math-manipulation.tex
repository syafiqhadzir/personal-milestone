\documentclass[10pt]{article}

\usepackage{amsmath, amsfonts, amssymb}
\usepackage{graphicx}

\begin{document}
	
	Display styled math puts math on display which is always centered.
	
	While, inline or text styled math stays in line.
	
	\[f(x) = (x + 2)^2-9\]
	
	Aligning multiple equations using align*
	
	\begin{align*}
		2x + 1 & = 9	& 3y - 2 & = -5		& -5z + 8 & =  3 \\
			2x & = 8	&	  3y & = -3		& 	  -5z & = -5 \\
			 x & = 4	&	   y & = -1		&		z & = -1
	\end{align*}

	\begin{align*}
		f(x) & = a_2 x^2 + a_1 x + a_0 \nonumber \\
			 & = x^2 + 4x - 5
	\end{align*}

	For inline, notice that by substitution we get the equation $f(x) = x^2 + 4x - 5$.
	
	Also, notice that by substitution we get the equation \(f(x) = x^2 + 4x - 5\).
	
	The variable \(x\).
	The letter x.
	
	More differences in the math modes,
	
	Display styled math,
	
	\(displaystyle \sum_{n=1}^{\infty}\frac{1}{n^2} = \frac{pi^2}{6} \)
	
	For inline,
	
	\(textstyle \sum_{n=1}^{\infty}\frac{1}{n^2} = \frac{pi^2}{6} \)
	
    
\end{document}